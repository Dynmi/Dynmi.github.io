\documentclass[letterpaper,11pt]{article}

%-----------------------------------------------------------
\usepackage[empty]{fullpage}
\usepackage{color}
\usepackage{hyperref}
\usepackage[UTF8]{ctex}
\usepackage{hyperref}
\hypersetup{
    colorlinks=true,
    linkcolor=blue,
    filecolor=blue,
    urlcolor=blue,
    citecolor=cyan,
}

\definecolor{mygrey}{gray}{0.80}
\raggedbottom
\raggedright
\setlength{\tabcolsep}{0in}

% Adjust margins to 0.5in on all sides
\addtolength{\oddsidemargin}{-0.5in}
\addtolength{\evensidemargin}{-0.5in}
\addtolength{\textwidth}{1.0in}
\addtolength{\topmargin}{-0.5in}
\addtolength{\textheight}{1.0in}

%-----------------------------------------------------------
%Custom commands
\newcommand{\resitem}[1]{\item #1 \vspace{-2pt}}
\newcommand{\resheading}[1]{{\large \colorbox{mygrey}{\begin{minipage}{\textwidth}{\textbf{#1 \vphantom{p\^{E}}}}\end{minipage}}}}
\newcommand{\ressubheading}[4]{
\begin{tabular*}{7.0in}{l@{\extracolsep{\fill}}r}
		\textbf{#1} & #2 \\
		\textit{#3} & \textit{#4} \\
\end{tabular*}\vspace{-6pt}}
%-----------------------------------------------------------


\begin{document}

\begin{tabular*}{7.5in}{l@{\extracolsep{\fill}}r}
\textbf{\large{Wang Han} }  & +86 13111866670 (cell)\\
SiChuan University, Shuangliu District& dynmi@foxmail.com\\
ChengDu, SiChuan 61027& https://github.com/Dynmi\\
\end{tabular*}
\\

\vspace{0.1in}


\resheading{Research Interests}
\\[9pt]
(Deep )Reinforcement Learning, \quad (Machine Learning )Computer Vison

\\[12pt]

\resheading{Education}
\begin{itemize}
\item
	\ressubheading{SiChuan University}{ChengDu}{Bachelor of Science in Computer Science}{Sep. 2018 - June. 2022 }
	\begin{itemize}
		\resitem{Currently at third year of Bechelor's degree}
	\end{itemize}
\end{itemize}



\resheading{Skills}

\begin{description}
\item[Languages:]
C/C{}\verb!++!, python, Linux Shell
\item[Operating Systems:]
Linux(especially experienced), Windows10
\item[Tools:]
Git, CUDA, LaTeX, Tensorflow2, Pytorch
\item[English level:]
CET4 549 points
\item[Mathematics:]
NEMT Math Mark 148 (full mark 150)
\end{description}


\resheading{Competition Experience}
\begin{itemize}
\item
	\ressubheading{Flower Classification with TPUs}{Kaggle}{Use TPUs to classify 104 types of flowers, hosted by Google Cloud}{2020}
	\begin{itemize}
		\resitem{Final rank top8\%}
	\end{itemize}
\item
	\ressubheading{Chinese Software Cup - College Student Software Design Competition }{National}{See our work here: \href{https://github.com/SCUCnSoftBei2020/SmartTraffic}{https://github.com/SCUCnSoftBei2020/SmartTraffic}}{2020}
	\begin{itemize}
		\resitem{Our Topic is "Traffic-Scene Application based on Computer Vision with Machine Learning Approches". We got National Silver Medal in the final round.}
	\end{itemize}
\item
	\ressubheading{Chinese National Undergraduate Mathematical Contest in Modeling}{National}{Problem B}{2020}
	\begin{itemize}
		\resitem{We got National Second Prize in the final round.}
	\end{itemize}
\item
	\ressubheading{Google CodeJam 2020}{}{End with Round1-C }{}
\end{itemize}


\resheading{Open Source Contributions}

\begin{description}
\item[Personal Tech Blog] \href{https://www.cnblogs.com/dynmi}{https://www.cnblogs.com/dynmi}
\item[Tensorflow] Active contributor to Tensorflow. See details here: \href{https://github.com/tensorflow/tensorflow/pulls?q=author\%3ADynmi+ }{https://github.com/tensorflow/tensorflow/pulls?q=author\%3ADynmi+ } 
\item[Pytorch] Volunteer to Pytorch official ch-version documention 
\end{description}

\resheading{Project Experience}
\begin{itemize}
\item
	\ressubheading{Implemention of AlexNet-7 }{\href{https://github.com/Dynmi/AlexNet7}{https://github.com/Dynmi/AlexNet7}}{}{}
	\begin{itemize}
		\resitem{Implemention of AlexNet-7, using C Program Language Without Any 3rd Library, according to the paper "ImageNet Classification with Deep Convolutional
Neural Networks" by Alex Krizhevsky,et al.}
	\end{itemize}
\item
	\ressubheading{N-Body Gravity Simulation}{\href{https://github.com/Dynmi/N-Body-Gravity-Simulation}{https://github.com/Dynmi/N-Body-Gravity-Simulation}}{}{}
	\begin{itemize}
		\resitem{Project finished with C Program Language as the final project for MIT6.S096 "Effective Programming in C and C++".}
	\end{itemize}
\item
	\ressubheading{Captcha Recognition}{\href{https://github.com/Dynmi/Captcha\_Recognition}{https://github.com/Dynmi/Captcha\_Recognition}}{}{}
	\begin{itemize}
		\resitem{} A fast and easy way to recognize captcha image, the cnn-model only has 5 layers, with an accuracy of 90\%.
	\end{itemize}
\item
	\ressubheading{Lian-Lian-Kan Game}{\href{https://github.com/L-W-X-X/Superior\_LianLianKan}{https://github.com/L-W-X-X/Superior\_LianLianKan}}{}{}
	\begin{itemize}
		\resitem{} This is the final project of our C++ course in SiChuan University. We used QT5 to implement this simple game.
	\end{itemize}

\end{itemize}


\resheading{College Experience}
\begin{itemize}
\item
	\ressubheading{Yoga Association of Sichuan University}{}{President}{April,2019 - July,2020}
	\begin{itemize}
		\resitem{}
	\end{itemize}
\item
	\ressubheading{Dance Association of Sichuan University}{}{Major Organizer}{Oct,2018 - Jun,2019}
	\begin{itemize}
		\resitem{}
	\end{itemize}

\end{itemize}

\resheading{Award}

\begin{description}
\item[Award of Everest Project ]
 Award to top10\% students in SiChuan University
\item[National Second Prize ]
 Chinese Software Cup College Student Software Design Competition 
\item[National Second Prize ]
 Chinese National Undergraduate Mathematical Contest in Modeling
\end{description}

\resheading{Hobbies}

\\[9pt]
Writing technology blogs, \quad Breaking-move Dance, \quad Chinese History Studies, \quad Buddhist Studies
\\[12pt]

\end{document}
